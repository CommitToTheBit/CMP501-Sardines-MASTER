% LaTeX Template for short student reports.
% Citations should be in bibtex format and go in references.bib
\documentclass[a4paper, 10pt]{article}
\usepackage[top=3cm, bottom=3cm, left = 2cm, right = 2cm]{geometry} 
\geometry{a4paper} 
\usepackage[utf8]{inputenc}
\usepackage{textcomp}
\usepackage{graphicx} 
\usepackage{amsmath,amssymb}  
\usepackage{bm}  
\usepackage[pdftex,bookmarks,colorlinks,breaklinks]{hyperref}  
\hypersetup{linkcolor=black,citecolor=black,filecolor=black,urlcolor=black} % black links, for printed output
\usepackage{memhfixc} 
\usepackage{pdfsync}  
\usepackage{fancyhdr}
\usepackage{fancyvrb}
\usepackage{natbib}
\usepackage{url}
%\pagestyle{fancy}
\usepackage{tabto}
\usepackage[shortlabels]{enumitem}
\usepackage{tikz}
\usepackage{verbatim}
\usepackage{nameref}
\usepackage{caption}
\usepackage{subcaption}

\begin{document}
\graphicspath{{./Images/}}
\begin{flushleft}

\section*{Sardines: A Networked Game}

\textit{Sardines} is a system of communication for 2-8 players. Each player controls a submarine in real time, viewing their surroundings through the lens of a retro sonar system. These submarines communicate in morse code, firing soundwaves - representative of dots and dashes - to one another across the map. While this system has ultimately been designed for integration into a larger, more complete game, \textit{Sardines} presents itself as a sandbox, to best restrict attention to the networking techniques at play.

\subsection*{Architecture}\label{Architecture}

\textit{Sardines}' network uses a straightforward client-server architecture, providing a single `master' game state essential in minimising disputes (see `\nameref{Prediction}'). Furthermore, the architecture uses local input processing - each client takes responsibility for processing their own player's actions and transmitting the resulting changes, rather than leaving the central server to compute a new master state directly from raw input. This distributes the game's physics-based movement calculations in a far more even fashion; an authoritative server with less trust in the player would better prevent cheating \citep{gmbta10}, but any notion of `cheating' will be of little concern while \textit{Sardines} remains at sandbox game.
%Reasons for choice...
%Though the current system does not have many variables to keep track of, the server also provides a `master state' for clients to work from.
%Citation about networking [peer-reviewed]...

\vspace{5pt}\noindent
\citeauthor{bauer04} \citeyearpar{bauer04} evaluate the scalability of this architecture, finding that with $n$ entity states (for the purposes of the report, players), client-server costs grow at order $O(n^2)$, compared to peer-to-peer's $O(n^3)$. They ultimately conclude that ``The client-server architecture exhibits the lowest growth in overall system cost, however, with the disadvantage that the entire growth must be handled by the central server.''  

\vspace{5pt}\noindent
In light of their findings, this report proposes a hybrid architecture for \textit{Sardines} to adopt at scale. While it ultimately proved too ambitious for this assignment, the original idea for the game was that \textit{multiple} players work together in piloting a singe submarine: a co-operative exercise with a navigator relaying key information about the surroundings, and a small team of other crewmates individually controlling acceleration, steering, etc. With only navigators witnessing the global game state first-hand, Figure \ref{Hybrid Architecture} positions these clients as `local' servers, processing (up to, say) $3$ players' inputs simulataneously, and relaying major changes to the game state back down to this crew. Not only does this topology require less of the central server's bandwidth (it still maintains $2-8$ connections, while each navigator has up to $4$ and other crewmates exactly $1$),  but demands far less calculation than if all $32$ players send their highly-individualised updates directly to the master state.

\begin{figure}[h]
\centering
\caption{Potential hybrid architecture for a $32$-player version of \textit{Sardines}.}
\label{Hybrid Architecture}
\end{figure}

\subsection*{Protocols}

\paragraph{Transport Layer}

TCP: Reliable, etc...
As much is set out in RFC 798 \citeyearpar{rfc793}, where \citeauthor{rfc793} introduces the protocol:.... %`` ''
Provide reliability paragraph?...

\vspace{5pt}\noindent
There is, of course, an argument for updating positions via UDP. Suppose that a submarine starts at position $A$, fails to update to position $B$: where the ordered nature of TCP means the application will resend and resend this update [EXPLAIN DOWNSIDE], this protocol will simply [GO TO the position C of the next successful].  While the deliberate simplicity of UDP - the [FEATURE \#1], the [FEATURE \#2] - lends itself to the continuous, incremental nature of movement , \textit{Sardines} will not make use of it. Submarines in this game travel slowly, with positions needed reliably but not immediately, and with the robustness of the prediction system used (see `\nameref{Prediction}'), updating position via TCP every 0.1s should be infrequent enough to avoid the backlog described above.\footnote{Could simplify even further with event-based - have server calculate all positions from key presses...}

\paragraph{Application Layer} 

While it doesn't consider the internetwork and hardware layers of the protocol stack, the application layer of \textit{Sardines}' network necessarily interacts with the transport layer directly below it. Client and server communicate using an underlying \texttt{Packet} struct, ...
[Serialisation!]...

\vspace{5pt}\noindent
At a more granular level, \textit{Sardines}' \texttt{SendablePacket}s may use any such struct as a \texttt{serialisedBody}:
\begin{itemize}[noitemsep]
\item \texttt{SyncPacket} Contains a \texttt{long} \texttt{syncTimestamp}. Sent on connection to standardise server and client calculations for \texttt{DateTime.UtcNow.Ticks} (which may vary from underlying OS to underlying OS, \citealp{msftUTC}).
\item \texttt{IDPacket} ... (in the hybrid architecture discussed above, this IP would also be used to establish crew-to-navigator connections).
\item \texttt{SubmarinePacket} 
\item \texttt{PositionPacket}
\item \texttt{MorsePacket}
\item \texttt{EmptyPacket} Contains no variables. Sent when the \texttt{bodyID} corresponds to a function with no arguments (e.g. when a client starts a game with \texttt{bodyID}.
\end{itemize}
\texttt{bodyID}s use the following naming convention: IDs \texttt{1XXX} correspond to clients connecting to/disconnecting from a server, \texttt{2XXX} to server functionality while in lobby mode, \texttt{3XXX} to server functionality while in a game mode, and \texttt{4XXX} to client actions which in-game.

\vspace{5pt}\noindent
While there isn't the space to break down every protocol in precise detail, consider the process of a player joining a lobby:
\begin{enumerate}[label=\textit{\arabic*}\textit{.}, noitemsep]
\item \textit{The client registers a TCP connection with the server.} The client then constructs... with ID \texttt{1000}...
\item \textit{The server receives a \texttt{SyncPacket} from the client.} Calling \texttt{Receive1000()}, the server...
\item \textit{The client receives a \texttt{SyncPacket} from the server.}
\item \textit{The server receives an \texttt{IDPacket} from the client.}
\item \textit{The client receives an \texttt{IDPacket} from the server.}
\item \textit{All clients receive (further) \texttt{IDPacket}s from the server.}
\end{enumerate}
There is arguably some unnecessary back-and-forth to the above, but small packet sizes shouldn't put any meaningful strain on bandwidth. The process is designed first and foremost for ease of programming: treating major protocols as a chain of smaller, simpler steps, it becomes far easier to manage - and document - the application layer. 
%This description may seem fairly dry, but it serves to ...
%There is some degree of back-and-forth to this process...


\subsection*{API}

% CHECKME: What is an API?
\textit{Sardines} is built with C\# in the Godot engine. It uses System.Net.Sockets to handle networking, and System.Runtime.InteropServices to serialize/deserialize packet structs. As noted in Microsoft's documentation \citeyearpar{msftSNS}, System.Net.Sockets implements conventional Berkeley sockets.




\subsection*{Integration}

Asynchronous I/O...
Connection class...

\vspace{5pt}\noindent
Discuss: offline vs. online updates to position!

\vspace{5pt}\noindent
This report would be amiss to skip over the final step: 

\subsection*{Prediction}\label{Prediction}

As discussed under `\nameref{Architecture}',  clients only send position updates every $0.1s$. What this report has so far failed to consider is how this appears to other clients - they experience what should be a smooth, continuous movement as discontinuous jumps over $0.1s$ intervals! Clearly, ... [introduce prediction - with reading?].

\vspace{5pt}\noindent
When a player chooses to move forward, they do not jump to a constant speed but continuously accelerate from zero; naturally, \textit{Sardines} uses second-order quadratic prediction to best approximate the second-order derivative of accelaration. Given a submarine's three most recent positions $\mathbf{r}_0$, $\mathbf{r}_1$,$\mathbf{r}_2$ (corresponding to times $t_0 > t_1 > t_2$), clients can average the velocities from $\mathbf{r}_1$ to $\mathbf{r}_0$, from $\mathbf{r}_2$ to $\mathbf{r}_1$, and the acceleration from $\mathbf{r}_1$ to $\mathbf{r}_0$ as
$$\mathbf{u}_0 = \frac{\mathbf{r}_0-\mathbf{r}_1}{t_0-t_1}, \;\; \mathbf{u}_1 = \frac{\mathbf{r}_1-\mathbf{r}_2}{t_1-t_2}, \;\; \mathbf{a}_0 = \frac{\mathbf{u}_0-\mathbf{u}_1}{t_0-t_1}, \;\; \textrm{respectively.}$$
These estimates define the quadratic model
$$ \mathbf{\tilde{r}}(t) = \mathbf{r}_0+\mathbf{u}_0t+\mathbf{a}_0t^2.$$
%FIXME: Subject to experiementation!
In contrast, the rudder controlling a submarine's rotation $\theta$ \textit{is} controlled at a constant speed, so \textit{Sardines} only uses linear prediction to approximate
$$\mathbf{\tilde{\theta}}(t) = \mathbf{\theta}_0+\mathbf{\dot{\theta}}_0t$$
(with subtle, case-specific considerations made given $\theta \in [0,2\pi)$).


\vspace{5pt}\noindent
If prediction is the act of waiting for data, then integration is how one `catches up' on receiving it. On receiving a new \texttt{PositionPacket} at time $t_0$, a programmer might be inclined to start predicting under to a new quadratic model $\mathbf{\tilde{r}}_{\textrm{new}}(t)$ immediately, but if positions $\mathbf{\tilde{r}}_{\textrm{old}}(t_0)$ and $\mathbf{\tilde{r}}_{\textrm{new}}(t_0)$ are visibly far apart, then the player will see the corresponding submarine make an instantaneous jump across the screen.\footnote{This might be regarded interpolation over $T = 0.0s$!} Instead, one takes a set time $T$ to linearly interpolate from the old trajectory to the new:
$$\mathbf{\tilde{r}}(t) = \begin{cases} 
\mathbf{\tilde{r}}_{\textrm{old}}(t) & \textrm{if $t < t_0$} \\
(1-q(t))\mathbf{\tilde{r}}_{\textrm{old}}(t) + q(t)\mathbf{\tilde{r}}_{\textrm{new}}(t) & \textrm{if $t_0 \leq t < t_0+T$} \\
\mathbf{\tilde{r}}_{\textrm{new}}(t) & \textrm{if $t \geq t_0+T$}
\end{cases}, \;\; \textrm{where $q(t) = \frac{1}{T}\left(t-t_0\right)$.}$$
In \textit{Sardines}' particular implementation, \texttt{PositionPacket}s are sent via TCP every $0.1s$; interpolation therefore takes place over a strictly shorter interval $T = 0.05s$. 
%Footnote - quirk of how we deal with being two packets behind?

\vspace{5pt}\noindent
[GRAPHICSX: Handdrawn diagram of interpolation and prediction interaction?]

\vspace{5pt}\noindent
To fully understand how \textit{Sardines} uses it prediction techniques, this report must first introduce a core challenge of any networked game: conflict resolution.

\vspace{5pt}\noindent
In \textit{Sardines}, the projectiles concerned are soundwaves. The visual language of the game, where soundwaves from external sources only become visible on collision with the player, provides a clear approach: the sender unequivocally takes precedence. Only when a player sees their soundwave hit another is a \texttt{MorsePacket} sent from their client (which will arrive with the usual delay). The sender knows with certainty who receives their message; the receiver, who cannot see the trajectory of the soundwave until it arrives, will have no sense of whether it ``should'' have hit them.

\vspace{5pt}\noindent
To further `smooth over' the application's conflict resolution, the receiving client makes use of backward prediction. Since neither server nor client stores more than three of any submarine's past positions at a time, it is fortunate the above formulae can approximate the past as well as the future.\footnote{\textit{Sardines}' submarines are physics-based objects, and at one point in development, the drag they experience was factored into prediction. However, the differential equations for 2D motion with a quadratic drag were too complex to find an analytic solution - rather than being able to substitute a $t$-value into a given equation, the prediction would be calculated over incremental, irreversible forward time steps - so the application sacrifices this more realistic model for the ability to look backwards in time.}

\vspace{5pt}\noindent
Suppose a sender emits a soundwave from position $\mathbf{r}$ at time $t_0$, which they see reach a receiver at $t_0+\Delta t$. On the arrival of the corresponding packet at $t_1$, then, the receiving client has to decide where the wave was emitted from \textit{in its local view of the game}. The obvious choice would be the `true origin' $\mathbf{r}$, but \textit{Sardines} uses the backwards prediction $\mathbf{\tilde{r}}(t_1-\Delta t)$. As [FIGURE] puts it in [REFERENCE], [QUOTE]; conflict resolution is the art of deciding which quantities are preserved across clients, and \textit{Sardines} - a system designed around slow, real-time communications - is far less concerned with a shared view of geography than it is a shared view of delay.

\subsection*{Testing}

[SORT THIS LAST THING - BUT PLAN THE TESTING OUT BY 15th?]

\bibliographystyle{agsm}
\bibliography{References}
\end{flushleft}
\end{document}

