% LaTeX Template for short student reports.
% Citations should be in bibtex format and go in references.bib
\documentclass[a4paper, 11pt]{article}
\usepackage[top=3cm, bottom=3cm, left = 2cm, right = 2cm]{geometry} 
\geometry{a4paper} 
\usepackage[utf8]{inputenc}
\usepackage{textcomp}
\usepackage{graphicx} 
\usepackage{amsmath,amssymb}  
\usepackage{bm}  
\usepackage[pdftex,bookmarks,colorlinks,breaklinks]{hyperref}  
\hypersetup{linkcolor=black,citecolor=black,filecolor=black,urlcolor=black} % black links, for printed output
\usepackage{memhfixc} 
\usepackage{pdfsync}  
\usepackage{fancyhdr}
\usepackage{fancyvrb}
\usepackage{natbib}
\usepackage{url}
%\pagestyle{fancy}
\usepackage{tabto}
\usepackage[shortlabels]{enumitem}
\usepackage{tikz}
\usepackage{verbatim}
\usepackage{nameref}
\usepackage{caption}
\usepackage{subcaption}

\begin{document}
\graphicspath{{./Images/}}
\begin{flushleft}

\section*{Sardines: A Networked Game}

[Introduction: explain concept of game]

\subsection*{Architecture}

\textit{Sardines} is built on a straightforward client-server architecture... 
Reasons for choice...
Though the current system does not have many variables to keep track of, the server also provides a `master state' for clients to work from.
Citation about networking [peer-reviewed]...

\vspace{5pt}\noindent
This may not remain true at a larger scale, however. While it ultimately proved too ambitious for this assignment, the original idea for . Since only navigators would be able to see the surrounding world, they could in some ways act as middle men in the hybrid architecture set out in Figure [FIGURE]. By treating each navigator as a `local' server for their crew of $4$, ...
This would, of course, be subject to further testing, as [carries issues of P2P?]...
[In paragraph/figure, introduce idea of navigator 'standardising' actions of the crew...]

\vspace{5pt}\noindent
[GRAPHICSX: Handdrawn diagram of interpolation and prediction interaction?]

\subsection*{Protocols}

\paragraph{Transport Layer}

TCP: Reliable, etc...
As much is set out in RFC 798 %\cite
, where John Postel introduces the protocol:.... %`` ''
Provide reliability paragraph?...

\vspace{5pt}\noindent
Why not consider UDP?...
Position updates every 0.1s...\footnote{Could simplify even further with event-based - have server calculate all positions from key presses...}

\paragraph{Application Layer}

Data sent with \texttt{Packet} struct...
Break down serialisation...

\vspace{5pt}\noindent
While there isn't the space to break down every protocol in precise detail... [List packets used and what each 
\begin{itemize}
\item
\end{itemize}

\vspace{5pt}\noindent
As an example, consider how a player might join the lobby...

\subsection*{API}

% CHECKME: What is an API?
\textit{Sardines} is built with C\# in the Godot engine. It uses System.Net.Sockets to handle networking, and System.Runtime.InteropServices to serialize/deserialize packet structs. This report notes that...



\subsection*{Integration}

Asynchronous I/O...
Connection class...

\vspace{5pt}\noindent
Discuss: offline vs. online updates to position!

\subsection*{Prediction}

Discussion of linear/quadratic prediction...


\vspace{5pt}\noindent
Technical breakdown of quadratic...
* End on note of limitations!

\vspace{5pt}\noindent
Discussion of interpolation - period is half of...

\vspace{5pt}\noindent
[GRAPHICSX: Handdrawn diagram of interpolation and prediction interaction?]

\vspace{5pt}\noindent
To fully understand how \textit{Sardines} uses it prediction techniques, this report must first introduce a core challenge of any networked game: conflict resolution.

\vspace{5pt}\noindent
In \textit{Sardines}, the projectiles concerned are soundwaves. The visual language of the game, where soundwaves from external sources only become visible on collision with the player, provides a clear approach: the sender unequivocally takes precedence. Only when a player sees their soundwave hit another is a \texttt{MorsePacket} sent from their client (which will arrive with the usual delay). The sender knows with certainty who receives their message; the receiver, who cannot see the trajectory of the soundwave until it arrives, will have no sense of whether it ``should'' have hit them.

\vspace{5pt}\noindent
To further `smooth over' the application's conflict resolution, the receiving client makes use of backward prediction. Since neither server nor client stores more than three of any submarine's past positions at a time, it is fortunate the above formulae can approximate the past as well as the future.

\vspace{5pt}\noindent
Suppose a sender $i$ emits a soundwave from position $\mathbf{r}_i$ at time $t_0$, which they see reach a receiver $j$ at $t_0+\Delta t$. On the arrival of a packet at $t_1$, then, the receiving client have to decide where the wave was emitted from \textit{in its local view of the game}. The obvious choice would be the `true origin' $\mathbf{r}_i$, but \textit{Sardines} uses the backwards prediction $\mathbf{\tilde{r}}_i(t_1-\Delta t)$. As [FIGURE] puts it in [REFERENCE], [QUOTE]; conflict resolution is the art of deciding which quantities are preserved across clients, and \textit{Sardines} - a system designed around slow, real-time communications - is far less concerned with a shared view of geography than it is a shared view of delay.

\subsection*{Testing}

[SORT THIS LAST THING - BUT PLAN THE TESTING OUT BY 15th?]

%\bibliographystyle{agsm}
%\bibliography{References}
\end{flushleft}
\end{document}

